\section{Introduction}

\subsection{Motivation}

In computational fluid dynamics (CFD) the finite volume method (FVM) is most commonly used to solve the partial differential equations (PDE) which describe the fluid flow. In order to discretize the simulation domain a mesh is constructed which covers the whole domain. The equations are then evaluated at the centroid of each cell and the results, such as velocity, pressure, temperature, etc are stored for each cell. In addition to the FVM some parts of a simulation may be modeled as particles such as Diesel droplets in an engine. \cite{nordin00} The gaseous phase, represented in the mesh, is called \emph{Eulerian}, while the liquid phase, represented by particles, is called \emph{Lagrangian}. Those are different ways of looking at the flow field. In the Lagrangian view the observer follows an individual particle as it moves through space and time. This can be visualized as sitting in a boat and drifting down a river. The Eulerian specification of the flow field is a way of looking at fluid motion that focuses on specific locations in space through which the fluid flows as time passes. This can be visualized by sitting on the bank of a river and watching the water pass the fixed location. \cite{eulerianLagrangian} There are different levels of coupling between these two phases: If the Lagrangian particles are just dragged by the Eulerian phase one refers to this as a \emph{one-way coupling}. If the Lagrangian particles also influence the Eulerian phase it is called a \emph{two-way coupling}. The mesh plays an important role when the two phases are coupled: The Eulerian equations are evaluated at the centroid of every cell. The Lagrangian particles do not need a mesh themselves, but the coupling requires it to know in which cell a particle resides so that the coupling terms can be appended to the Eulerian phase. \emph{More specifically we need to know for every time step through which cells the particle travels and how much time it spends in each cell.}

It is distinguished between structured and unstructured meshes. Structured refers to the way how the mesh data is stored in memory: A direct mapping between the connections in the mesh and the addresses of the data exists. This structure restricts the shape of the cells in the mesh to hexahedra. Unstructured meshes on the other hand allow cells with any number of faces, this makes them more flexible but also increases the amount of work a program has to do, for example to access a cell next to a given cell. Structured meshes can be transformed into a uniform cartesian grid. \cite{innovativeCFDGrid} It is therefore trivial to find the cell in which a particle resides, given its position. In an unstructured mesh this is no longer possible and requires a sophisticated algorithm which is explained in section \ref{sec:particleTrackingAlgo}. OpenFOAM comes with support for unstructured meshes because this makes it simpler to automatically mesh complex geometries from a CAD system and it simplifies the import of meshes from different meshing tools. \cite{jasakMeshHandling}

\subsection{Project Goal}

Particle tracking requires the execution of simple vector calculations for a large number of particles. This can be achieved more efficiently by using a massively parallel processor such as a GPU (see section \ref{sec:gpuComputing}).
Based on the existing OpenFOAM code the goal is to develop a solver which moves particles using the velocity field from the mesh. The solver will use the GPU to execute the computations required for the particle tracking algorithm in a massively parallel way, which will lead to great speedup compared to the already existing single-threaded implementation in OpenFOAM. The original project definition can be found in the appendix, section \ref{sec:projectDef}.
