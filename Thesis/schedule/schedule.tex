\documentclass[
  a4wide,
  smallheadings
]{article}

\usepackage{paralist}

\usepackage{a4wide}
\usepackage{supertabular}

\begin{document}
    
%\section*{Project Schedule}

\begin{supertabular}{|p{1.5cm}|p{2cm}|p{2cm}|p{8cm}|}
    
    \hline 
    Week Nr. & Monday & Friday & Tasks \\
    
    \hline \hline
    
    8 & 21. Feb & 25. Feb &

   \begin{compactitem}
      \item Create Latex template, outline and title page. 
      \item Put together reference list (Bibtex).
    \end{compactitem}\\
    
    \hline
    
    9 & 28. Feb & 04. March &

   \begin{compactitem}
        \item Start with creating a solver, based on icoLagrangianFoam in OF 1.5 ext.
        \item Figure out unit testing and integrate googles gtest framework.
    \end{compactitem}\\

    \hline

    10 & 07. March & 11. March &
    

   \begin{compactitem}
        \item Study the OF cloud classes and figure out which one can be used for
              a simple one-way coupling.
        \item Write the project schedule and hand it in.
    \end{compactitem} \\

    \hline

    11 & 14. March & 18. March &
    

    \begin{compactitem}
        \item Pack the mesh data so that it can be moved to the GPU. Write tests
              and validate.
        \item Do this with regard to performance, the data should be algined in
              a way that requires a min amount of read write access to do the
              tracking.
        \item Think the whole particle tracking process through, create a flow
              chart.
        \item Learn Doxygen.
    \end{compactitem} \\

    \hline

    12 & 21. March & 25. March &
    

    \begin{compactitem}
        \item Pack all the particle data (position, velocity, etc) for the GPU
              and validate.
        \item Find relevant literature about CFD basics.
        \item Reread basic CFD and start working on the CFD Basics section for
              the documents.
    \end{compactitem} \\
    
    \hline

    13 & 28. March & 01. April &
    

    \begin{compactitem}
        \item Get the first very basic case working where the tracking algo is
              executed on the GPU.
        \item Finish first draft of CFD basics.
    \end{compactitem} \\
    
    \hline
    
    14 & 04. April & 08. April &
    

    \begin{compactitem}
        \item Estimate how much time is spent for moving data between GPU and
              CPU memory. How much faster is the calculation?
        \item Start writing about the Eulerian-Lagrangian coupling.
    \end{compactitem} \\

    \hline

    15 & 11. April & 15. April &
    

    \begin{compactitem}
        \item Work on the demo cases (1D tunnel, torus). Is there need for
              another case, maybe with more complicated cells?
        \item Identify problems with the existing implementation, think of
              improvement.
    \end{compactitem} \\
    
    \hline
    
    16 & 18. April & 22. April &
    

    \begin{compactitem}
        \item Based on the observations and exerience so far start rewriting
              parts of the solver if needed.
        \item Create diagrams and explain the implementation.
    \end{compactitem} \\
    
    \hline
    
    17 & 25. April & 29. April &
    

    \begin{compactitem}
        \item Work on the documentation. Write the OpenFOAM part, based on the 
              Semesterproject. (Mesh representation in OF, most important classes
              basic data structures, such as lists and linked lists, etc)
    \end{compactitem} \\
    
    \hline
    
    18 & 02. May & 06. May &
    

    \begin{compactitem}
        \item Documenation: Add the part about GPU computing.
        \item Finish a first draft and give it to somebody to read.
    \end{compactitem} \\
    
    \hline
    
    19 & 09. May & 13. May &
    

    \begin{compactitem}
        \item 
    \end{compactitem} \\
    
    \hline
    
    20 & 16. May & 20. May &
    

    \begin{compactitem}
        \item
    \end{compactitem} \\
    
    \hline
    
    21 & 23. May & 27. May &
    

    \begin{compactitem}
        \item
    \end{compactitem} \\
    
    \hline
    
    22 & 30. May & 03. June &
    

    \begin{compactitem}
        \item
    \end{compactitem} \\
    
    \hline
    
    23 & 06. June & 10. June &
    

    \begin{compactitem}
        \item
    \end{compactitem} \\
    
    \hline
    
    24 & 13. June & 17. June &
    

   \begin{compactitem}
        \item
    \end{compactitem} \\
    
    \hline

\end{supertabular}
    
\end{document}