\section{Conclusion and Future Work}

It was shown that the Lagrangian particle tracking algorithm can run efficiently on the GPU. Porting it required the conversion of the mesh data into structures suitable for the SIMT architecture. The algorithm itself could be parallelized easily since the computations are done for each particle individually which is well suitable for massively parallel processors. The biggest problem, with respect to efficiency, are the time consuming memory copies over the PCI bus. Having to copy data over a slow bus repeatedly can nullify speedups achieved by porting parts of a simulation to massively parallel hardware. While there exist several workarounds, such as the possibility to overlap copying with kernel execution, the problem will probably be gone in future hardware generations. For example Advanced Micro Devices (AMD) already announced their next generation architecture called ``Fusion'' \cite{fusionWhitepaper} in which the GPU and the CPU will access memory over the same high speed bus.

Now that the basics for Lagrangian particles on GPUs are implemented, using it in an actual simulation is the next step. Just executing the tracking on the GPU would probably not have a big impact on the overall run time, mostly due to the fact that the particle data must be converted into a suitable format and copied over a slow bus for each time-step\footnote{It is interesting to see how other teams dealt with this problem. For example a large CFD code written in Fortran code using OpenMP was automatically translated into CUDA code. Having the whole code running on the GPU eliminated the need to copy data between two address spaces during the simulation. It is only required to upload the data at the beginning and download the results at the end. \cite{portingCuda10} (See section \ref{sec:cudaPorting10} for a summary.) With OpenFOAM this would be unfeasible because of the lack of fine-grained parallelism, the immense complexity of the C++ code and the unsuitable data structures for massively parallel processing.}. However if there are other computationally intense tasks involving particles such as collision detection significant improvements in the overall run time could be achieved. In 2000 Niklas Nordin wrote in his thesis about Diesel combustion \cite[Chapter 2.2.6]{nordin00} ``Among the spray sub-models, the weakest model is the collision model.'' Reasons for this are the mesh dependency of the collision model and the fact that some collision models do not even take the trajectory of the particles into account. Collision models which work independently of the mesh and use the trajectories of the particles tend to be computationally intensive because each particle must be collision checked with a subset of all particles. Nvidia already showed that collisions can efficiently be calculated on a GPU (see section \ref{sec:summaryNvParticles}). Adding such collision models would greatly improve the Lagrangian framework.